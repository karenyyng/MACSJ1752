%\pdfoutput=1
%\input{style.tex}
\documentclass[letterpaper,useAMS,usenatbib]{mn2e}
\usepackage[colorlinks=true,
            linkcolor=blue,
            urlcolor=blue,
					  citecolor=blue]{hyperref}
\usepackage{amssymb}
\usepackage{graphicx}
\usepackage{amsmath}
\usepackage[amssymb]{SIunits} 
\usepackage{booktabs}
\usepackage{hhline}
\usepackage{breqn}
\usepackage{standalone}
\usepackage{dcolumn}
	\newcolumntype{d}[1]{D{.}{.}{#1}}
\usepackage{tabularx}
\usepackage{booktabs}
\usepackage{microtype}
\graphicspath{{graphics/}}
\newcommand{\mc}[1]{\multicolumn{1}{c}{#1}} % handy shortcut macro
%-----------------------------------------------------------------------
\def\apjl{ApJL }
\def\aj{AJ }
\def\apj{ApJ }
\def\pasp{PASP }
\def\spie{SPIE }
\def\apjs{ApJS }
\def\araa{ARAA }
\def\aap{A\&A }
\def\nat{Nature }
\def\mnras{MNRAS }
\def\mnrasl{MNRASL }
\providecommand{\eprint}[1]{\href{http://arxiv.org/abs/#1}{#1}}
\providecommand{\adsurl}[1]{\href{#1}{ADS}}
\providecommand{\ISBN}[1]{\href{http://cosmologist.info/ISBN/#1}{ISBN: #1}} 
%-----------------------------------------------------------------------
\title[
	Awesome analysis of MACS1752 
]
{Awesome analysis of MACS1752}
\author[Karen Y. Ng et al.]{Karen Y. Ng,$^{1}$
	[order TBD]
	M. James Jee,$^{2}$
	D. Wittman,$^{1}$
	William A. Dawson,$^{3}$ 
	\newauthor Nathan Golovich$^{1}$
}
\begin{document}
\date{arXiv} \pagerange{\pageref{firstpage}--\pageref{lastpage}}
\pubyear{2015} \maketitle\label{firstpage}
\begin{abstract} 
	
\end{abstract}
\begin{keywords}
Galaxies: clusters: individual: MACS J1752.0+4440; Large-scale structure of
Universe; methods: statistical 
\end{keywords}
\section{INTRODUCTION} 
\section{DATA}
\subsection{Keck DEIMOS Observations and Spectra Reduction}
\begin{itemize}
	\item Date and observation conditions
	\item Refer to Will's 2014 paper for details of spectra reduction 
\end{itemize}

\subsection{Suburu / SuprimeCam Observation}
Date and observation conditions

\subsection{Hubble Space Telescope Observation} 
Date and observation conditions

\subsection{Data Reduction for Subaru and HST data}
refer to James' CIZA paper

\subsubsection{Extinction Correction}
There is low extinction in the field of view of MACSJ1752. The mean of the extinction
magnitude in E(B-V) is 0.036, while the corresponding standard deviation 
is $8.5 \times 10^{-3}$ (ACCORDING TO WHOSE MAP?).
We first perform cubic interpolation to infer the E(B-V) values at the spatial
location of all the entries of our data catalog. Then we perform dust
correction for other bands according to Schlafly \& Finkbeiner (reference from CIZA paper 3.3).

\subsubsection{Shape measurement and source selection}
[I may just move all these to a new paper and stick with a quick-and-dirty
approach]
We make use of two state-of-the-art classification / regression algorithms,
gradient boosting trees and a Random Forest algorithm,
to select our source and obtained consistent results. 
Data cleaning, preprocessing and feature engineering have always been considered to
be one of the most important steps in a data analysis (quote Machine learning
paper). 

The predictors that we included in the classification algorithms are all the
colors and magnitudes that we have, the half-light radius, identification
status flag and the redshift. 
The class labels in the predictions are 
member galaxies, source galaxies, and other contaminants such as stars,
foreground galaxies, and unidentifiable contaminants with problematic status
flags.

To minimize misclassification rates, we discarded samples with classification
probability less than 0.5 in the dominant classification. 

We make use of stratefied cross validation to find a  
training error rate is 
test error rate is .

The false positive rate from the out-of-bag samples is .

After obtaining the source population, 
we perform signal-to-noise ratio (SNR) cuts to make sure the ellipticity errors are $< 0.3$ 
and the SNR are $>5$.

The resulting source density count from our cleaned catalog is , which is
within $\sigma$ of the source density count of the training COSMOS fields
catalog after removing stars and foreground galaxies.  
Finally, the effective lensing depth $\beta$, i.e. the mean of $D_{LS} / D{L}$[DOUBLE
CHECK] etc. of the inferred source population is .  


\begin{itemize}
	\item K-correction? (not needed)
	\item Selection based on $(g-i)$ vs i band? 
	\item Star galaxy separation based on half-light radius 
	\item Source density counts   
	\item S/N cuts - ellipticity error $< 0.3$ and detection significance $> 5\sigma$
\end{itemize}

\subsubsection{Source redshift estimation}
Cosmic Evolution Survey photometric catalog (COSMOS) 
Ilbert et al. 2009
comparison of depth between our Subaru image and the COSMOS image?

\section{METHOD}

\subsection{Optical analysis}
\subsubsection{Determining the number of galaxy subclusters and membership}
\subsubsection{Brightest Cluster Galaxies identification (BCG)}
\subsubsection{Number density and luminosity map}

% We make use of the Extreme Deconvolution Gaussian Mixture Model (XDGMM) to
% determine the number of subclusters. This algorithm allows us to 
% 1) get rid of the artificial imprint of the mask of the Keck
% DEIMOS spectrograph, 2) separate stars from galaxies, 3) separate
% foreground or background galaxies.  
% 
% The features that we make use of are in the XDGMM include the half-light radius, $I_{iso}$,
% $G_{iso}$, $R_{iso} - I_{iso}$, etc. 
% 
% It is customary to use more cluster than needed for high
% dimensional GMM then manually identify the relevant clusters (See Statistics,
% Data Mining and Machine Learning in Astronomy reference.)
% 
% We further perform a weighted Kernel Density Estimation (KDE) from the
% identified clusters (Ng et al. 2015) to examine the number density
% peaks and the luminosity peaks.

\subsubsection{Dynamics of the subclusters}
\begin{itemize}
		\item LOS velocities
		\item mass estimation from velocity dispersion 
\end{itemize}
% We continue to make use of the identification of cluster members from the XDGMM
% in this analysis, but we only make use of the member galaxies with secure
% spectroscopic redshift. 
% We use the biweight statistic to determine the location and the scale of
% the relative line-of-sight (LOS) velocities of the subclusters. A previous,
% less extensive use of the GMM has been shown by the ECGMM people.

\subsection{Weak lensing (WL) analysis with LENSTOOL}
% To determine the source redshift of the lensed galaxies, we train a Random
% Forest regressor based on the data in the COSMOS field. 


\subsection{Offset between the DM and galaxy centroids}

\subsection{Setup of Dawson's dynamical simulation}
\subsubsection{Weights due to radio relic info}

\section{RESULTS}

% We cross validate our results by withholding 10\% of our source galaxies. 



\section{DISCUSSION}
\subsection{Offset between the DM centroids and galaxy centroids}
\section{ACKNOWLEDGEMENTS}
% KN would like to thank Phil Marshall and Jake VanderPlas for their advice on the use of GMM and
% XDGMM during the AstroData Hack week 2014.

\bibliographystyle{mn2e}
\bibliography{MACS1752.bib}
\appendix
\section{MCMC diagnostics from the WL analysis}
Fig 1. Chains indicating burn-in and the posterior density 
acceptance rate!


\section{Outputs from dynamical simulation}
\clearpage\bsp\label{lastpage} 
\end{document}
