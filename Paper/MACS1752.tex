%\pdfoutput=1
%\input{style.tex}
\documentclass[letterpaper,useAMS,usenatbib]{mn2e}
\usepackage[colorlinks=true,
            linkcolor=blue,
            urlcolor=blue,
					  citecolor=blue]{hyperref}
\usepackage{amssymb}
\usepackage{graphicx}
\usepackage{amsmath}
\usepackage[amssymb]{SIunits} 
\usepackage{booktabs}
\usepackage{hhline}
\usepackage{breqn}
\usepackage{standalone}
\usepackage{dcolumn}
	\newcolumntype{d}[1]{D{.}{.}{#1}}
\usepackage{tabularx}
\usepackage{booktabs}
\usepackage{microtype}
\graphicspath{{graphics/}}
\newcommand{\mc}[1]{\multicolumn{1}{c}{#1}} % handy shortcut macro
%-----------------------------------------------------------------------
\def\apjl{ApJL }
\def\aj{AJ }
\def\apj{ApJ }
\def\pasp{PASP }
\def\spie{SPIE }
\def\apjs{ApJS }
\def\araa{ARAA }
\def\aap{A\&A }
\def\nat{Nature }
\def\mnras{MNRAS }
\def\mnrasl{MNRASL }
\providecommand{\eprint}[1]{\href{http://arxiv.org/abs/#1}{#1}}
\providecommand{\adsurl}[1]{\href{#1}{ADS}}
\providecommand{\ISBN}[1]{\href{http://cosmologist.info/ISBN/#1}{ISBN: #1}} 
%-----------------------------------------------------------------------
\title[
	Awesome analysis of MACS1752 
]
{
	Awesome analysis of MACS1752
}
\author[Karen Y. Ng et al.]{Karen Y. Ng,$^{1}$
	[order TBD]
	M. J. James,$^{2}$
	D. Wittman,$^{1}$
	William A. Dawson,$^{3}$ 
	\newauthor Nathan Golovich,$^{1}$
}
\begin{document}
\date{arXiv} \pagerange{\pageref{firstpage}--\pageref{lastpage}}
\pubyear{2015} \maketitle\label{firstpage}
\begin{abstract} 
	
\end{abstract}
\begin{keywords}
Galaxies: clusters: individual: MACS J1752.0+4440; Large-scale structure of
Universe;
\end{keywords}
\section{INTRODUCTION} 
\section{DATA}
\section{METHOD}


\subsection{Number of subcluster components from optical analysis}
We make use of the Extreme Deconvolution Gaussian Mixture Model (XDGMM) to
determine the number of subclusters. This algorithm allows us to 
1) get rid of the artificial imprint of the mask of the Keck
DEIMOS spectrograph, 2) separate stars from galaxies, 3) separate
foreground or background galaxies.  

The features that we make use of are in the XDGMM include the half-light radius, $I_{iso}$,
$G_{iso}$, $R_{iso} - I_{iso}$, etc. 

It is customary to use more cluster than needed for high
dimensional GMM then manually identify the relevant clusters (See Statistics,
Data Mining and Machine Learning in Astronomy reference.)

We further perform a weighted Kernel Density Estimation (KDE) from the
identified clusters (Ng et al. 2015) to examine the number density
peaks and the luminosity peaks.

\subsection{LOS velocities of the subclusters}
We continue to make use of the identification of cluster members from the XDGMM
in this analysis, but we only make use of the member galaxies with secure
spectroscopic redshift. 
We use the biweight statistic to determine the location and the scale of
the relative line-of-sight (LOS) velocities of the subclusters. A previous,
less extensive use of the GMM has been shown by the ECGMM people.

\subsection{Weak lensing analysis with LENSTOOL}
To determine the source redshift of the lensed galaxies, we train a Random
Forest regressor based on the data in the COSMOS field. 
\section{RESULTS}
\section{WL analysis}

We cross validate our results by withholding 10\% of our source galaxies. 



\section{Offset between the DM centroids and galaxy centroids}
\section{DISCUSSION}
\section{ACKNOWLEDGEMENTS}
KN would like to thank Phil Marshall and Jake VanderPlas for their advice on the use of GMM and
XDGMM during the AstroData Hack week 2014.

\bibliographystyle{mn2e}
\bibliography{MACS1752.bib}
\appendix
\section{MCMC diagnostics from the WL analysis}
\section{Outputs from dynamical simulation}
\clearpage\bsp\label{lastpage} 
\end{document}
